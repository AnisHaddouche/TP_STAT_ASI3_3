\documentclass{article}

\usepackage[utf8]{inputenc}
\usepackage[T1]{fontenc}
\usepackage[french]{babel}
\usepackage{graphicx}
\usepackage[margin=3cm]{geometry}
\usepackage{fancyhdr}
\usepackage{listings}

\lstset{
    breaklines=true,
    numbers=left,
    frame=single
    }

\pagestyle{fancy}
\fancyhead[L]{NOM(S) et Prénom(s) ---}
\fancyfoot{}

\begin{document}

\begin{center}
    {\Huge Analyse multidimensionnelle - Variables qualitatives}
\end{center}

\section{Les données}

Type des variables :
\begin{description}
    \item[Survived]
    \item[Pclass]
    \item[Sex]
    \item[Age]
    \item[SibSp]
    \item[Parch]
    \item[Fare]
    \item[Embarked]
\end{description}

\begin{table}[h] 
    \centering
    \begin{tabular}{l|c|c}
        Variable & Nombre de champs & Nombre de données manquantes\\ \hline
        & & \\ % Recopier cette ligne pour chaque variable qualitative
    \end{tabular}
\end{table}

\begin{figure}[h!] %Représentation graphique de SibSp
    \centering
    %\includegraphics[width=8cm]{image.png}
\end{figure}

\begin{figure}[h!] %Représentation graphique de Embarked
    \centering
    %\includegraphics[width=8cm]{image.png}
\end{figure}

\begin{figure}[h!] %Représentation graphique de Fare
    \centering
    %\includegraphics[width=8cm]{image.png}
\end{figure}

\newpage
\section{Tableaux de contingence}
\subsection{La survie selon le genre}

\begin{table}[h!]  % A remplir à la main, avant d'utiliser la fonction crosstab
    \centering
    \begin{tabular}{l|c|c}
         & Female & Male\\ \hline
        Survived & & \\
        Did not survived & & \\
    \end{tabular}
\end{table}



% Analyser ce que vous observez et proposer une hypothèse

\subsection{La survie selon la classe et le port d'embarquement}

\begin{table}[h!]  % A remplir avec la sortie de crosstab
    \centering
    \begin{tabular}{l|c|c|c}
        & Classe 1 & Classe 2 & Classe 3\\ \hline
        Survived & & & \\
        Did not survived & & & \\
    \end{tabular}
\end{table}

\begin{table}[h!]  % A remplir avec la sortie de crosstab
    \centering
    \begin{tabular}{l|c|c|c|c}
        & 67 & 81 & 83 & C\\ \hline
        Survived & & & & \\
        Did not survived & & & & \\
    \end{tabular}
\end{table}

% Formuler des hypothèses sur vos observations

\newpage
\section{Dépendance de variables}

Interprétation des valeurs des coefficients :
\begin{description}
    \item[$\chi^{2}$] 
    \item[$\Phi^{2}$] 
    \item[Tschuprow] 
    \item[Cramer] 
\end{description}

% Copier-coller les tableaux d'interdépendance obtenus dans Matlab pour chaque coefficient
% Chi2
\begin{lstlisting}[numbers=none] 
\end{lstlisting}

%Phi2
\begin{lstlisting}[numbers=none]
\end{lstlisting}

%Tschuprow
\begin{lstlisting}[numbers=none]
\end{lstlisting}

%Cramer
\begin{lstlisting}[numbers=none]
\end{lstlisting}

% Commenter vos hypothèses faites précédemment à l'aide de ces tableaux

\newpage


\begin{table}[h!]  % A adapter selon le nombre de modalités de chaque variable choisie
    \centering
    \begin{tabular}{l|c|c|c|c}
        &  &  &  & \\ \hline
        &  &  &  & \\
        &  &  &  & \\
    \end{tabular}
\end{table}


\section{Tableaux de Burt}

% Faites un copier-coller de votre tableau ou prenez une capture d'écran (au choix)
\begin{lstlisting}[numbers=none]
\end{lstlisting}

\begin{figure}[h!] %Représentation graphique des profils-colonnes précédents
    \centering
    %\includegraphics[width=8cm]{image.png}
\end{figure}

% A décommenter si vous arrivez là !

%\newpage 
%\section{Bonus pour les plus rapides}
%
%
%\begin{figure}[h!] %Représentation graphique de Age, avec classes d'amplitudes égales
%    \centering
%    %\includegraphics[width=8cm]{image.png}
%\end{figure}
%
%\begin{figure}[h!] %Représentation graphique de Age, avec classes d'amplitudes variables
%    \centering
%    %\includegraphics[width=8cm]{image.png}
%\end{figure}
%
%% Commenter ces deux graphiques
%
%% Explorer les relations entre les variables quantitatives des données

\newpage
\appendix

\section{Code}
%\lstinputlisting[language=Matlab]{sourceTP.m}

\end{document}