\documentclass[11pt]{exam}
\RequirePackage{amssymb, amsfonts, amsmath, latexsym, verbatim, xspace, setspace}
\usepackage[french]{babel}
\usepackage[utf8]{inputenc}
\usepackage{url}
\usepackage[french]{babel}
\usepackage{graphicx}
\usepackage{hyperref}
\usepackage[]{geometry}
\usepackage{booktabs}
%\geometry {portrait, includehead, left=2cm, right=2cm, top=2cm, bottom=2cm}

\hypersetup{
    unicode=false,          % non-Latin characters in Acrobat?s bookmarks
    colorlinks=true,       % false: boxed links; true: colored links
    linkcolor=red,          % color of internal links (change box color with linkbordercolor)
    citecolor=green,        % color of links to bibliography
    filecolor=magenta,      % color of file links
    urlcolor=magenta           % color of external links
}

\pagestyle{headandfoot}

\def\dbC{{\mathrm{I\hskip-4.7pt C}}}
\def\dbR{{\mathrm{I\hskip-2.2pt R}}}
\def\dbN{{\mathrm{I\hskip-2.2pt N}}}
\def\esp{{\mathrm{I\hskip-1.5pt E}}}
\def\pr{{\mathrm{I\hskip-2.2pt P}}}
\def\clL{{\mathrm{\cal L}}}
\def\cltL{{\mathrm{\widetilde{\cal L}}}}
\def\w{{\mathbf{w}}}
\def\x{{\mathbf{x}}}
\def\Blambda{{\mathbf{\lambda}}} 

% =================================================================================
\newcommand{\moncadre}[1]%
{\fbox{
    \begin{minipage}{400pt} #1
    \end{minipage}
}}

%\figgauche{LargeurImage}{FichierImage.eps}{TexteADroite}
%
\newlength\jataille
\newcommand{\figgauche}[3]%
{\jataille=1\linewidth\advance\jataille by -#1
      \advance\jataille by -.5cm
      \begin{minipage}[c]{6pc}
      \includegraphics[width=6pc]{#2}
      \end{minipage}\hfill
      \begin{minipage}[c]{30pc} #3
      \end{minipage}
		~\newline{}
}


% =================================================================================
\extraheadheight{50pt}
\extrafootheight{-70pt}
\firstpageheader{\includegraphics[width=100pt]{logoinsa} \\
  B. Gaüzère, \\ A. Rogozan}
{\Large \bf ~~TP \\ 
  Statistiques Descriptives \\ Analyse multidimensionnelle \\ de Données Qualitatives}
{\includegraphics[width=90pt]{logoasi} \\ $ 3^{\mbox{\footnotesize
      ème}}$ année}
\runningheader{ASI3}{Stats}{Analyse multi dimensionnell}
\headrule
\footer{}{}{p.\thepage/\numpages}

% Here's where you edit the Class, Exam, Date, etc.
\newcommand{\lesson}{Statistiques}
\newcommand{\chapter}{Statistiques descriptives}
\newcommand{\class}{ASI 3}
\newcommand{\semestre}{Semestre 5}
\newcommand{\authors}{B. Gauzere, A. Rogozan}

% For an exam, single spacing is most appropriate
% \singlespacing
% \onehalfspacing
% \doublespacing

% For an exam, we generally want to turn off paragraph indentation
\parindent 0ex
\newcommand{\matlab}[1]{ \verb|>>| #1 }
\begin{document} 

\section{Les données }
Les données que nous allons analyser dans ce TP sont extraites de \url{https://github.com/caesar0301/awesome-public-datasets} et
concerne les caractéristiques des passagers du Titanic. La description
des données originales est disponible sur cette page web :
\url{https://www.kaggle.com/c/titanic/data}. 

%Ce TP concerne l'analyse multidimensionnelle de données
%qualitatives. Par conséquent, les données originales ont été en parti
%retraitées afin d'inclure seulement des données qualitatives. Vous
%pouvez récupérez ces données en téléchargeant le fichier
%\verb|titanic-quali.csv| sur Moodle. 

%\begin{table}[h]
%  \centering
%  \begin{tabular}{lll}
%    \toprule
%    Champ & Libellé & Signification \\
%    \midrule
%    1 & Survie & 0 : Décédé, 1 : survivant \\
%    2 &  Classe &  Indique la classe du passager : 1,2 ou 3ème classe} \\
%    3 & Genre & 0 : féminin, 1 : masculin\\
%    4 & Age & 0 : Inconnu, 1 : bébé, 2 : adoslescent, 3 : adulte, 4 :
%              personne agée.\\
%    5 & Frères/s\oe urs à bord & 0 : aucun, 1 : Un, 2 : plus de
%                                      un \\
%    6 & Parents à bord & 0 : aucun, 1 : Un, 2 : plus de
%                                      un \\
%    7 & Port d'embarquement & 67 : Cherbourg, 81 : Queenstown, 83 :
%                              Southampton\\ 
%    \bottomrule
% \end{tabular}
%  \caption{Données du fichier titanic-quali.csv}
%  \label{tab:data}
%\end{table}

\begin{questions}
  \question Charger les données et visualisez les données. Controlez le
  nombre d'entrées, le nombre de champs et leur étendue.
  \question Enlever la variable "Fare".
  \question Recoder la variable "Sex" en variable binaire (male= 0, female=1).
  \question Remplacer les valeurs manquantes dans la variable age par 0.
  \question Transformer les variables "Sex" et "Embarked" en entier (int)
\end{questions}

\section{Tableaux de contingence}

\subsection{La survie selon le genre}
\begin{questions}
  \question Calculez le tableau de contingence sur la survie \emph{vs}
  le genre des passagers. 
%  Profitez de la qualité binaire des données
%  pour calculez le tableau de contingence par une multiplication de matrices.
  \question Analysez ce que vous observez et proposez une hypothèse.
\end{questions}

\subsection{La survie selon la classe et le port d'embarquement}
\begin{questions}
  \question Calculez le tableau de contingence sur la survie \emph{vs}
  la classe des passagers.

%  \emph{Aide :}
%  \begin{itemize}
%  \item \verb| >> help histc|
%  \end{itemize}
  \question Calculez le tableau de contingence sur la survie \emph{vs}
  le port d'embarquement.
  \question Formulez des hypothèses sur vos observations.
\end{questions}

\section{Dépendance de variables}

\begin{questions}
  \question Implémentez le calcul du $\chi^2$ et des mesures issues du
  $ \chi^2$ :
  \begin{itemize}
  \item $\Phi^2$,
  \item Coefficient de Tschuprow,
  \item Coefficient de Cramer.
  \end{itemize}
  \question Validez on invalidez vos hypothèses faites dans la section suivante.
  \question Pour chaque mesure , calculez un tableau présentant
  l'interdépendance pour l'ensemble des variables utilisées.
  \question Que déduisez vous sur vos hypothèses précédentes ? 
\end{questions}
%\section{Profils}
%\begin{questions}
%  \question Implémentez le calcul des profils-colonnes pour une paire
%  de descripteurs de votre choix. 
%  \question Affichez ces profils-colonnes sous forme de graphiques
%  (voir figure \ref{fig:profils}). Mettez en relation ce que
%  vous observez et les hypothèses que vous avez faites
%\end{questions}

%\begin{figure}[h]
%  \centering
%  \includegraphics[width=0.7\textwidth]{profils}
%  \caption{Profils-colonnes relatifs à la table de contingence de la
%    classe des passagers vs au port d'embarquement. }
%  \label{fig:profils}
%\end{figure}
\section{Tableaux de Burt}
\begin{questions}
  \question Calculez le tableau de Burt pour l'ensemble des variables
  et leurs modalités.
\end{questions}
\section*{Bonus}
\begin{questions}
  \question Pour les plus rapides, récupérez les données originales et
  effectuez une analyse plus poussée sur le profil des survivants en
  utilisant les données quantitatives.

\end{questions}
 \end{document}
%%% Local Variables:
%%% mode: latex
%%% TeX-master: t
%%% End:
